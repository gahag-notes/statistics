\documentclass{article}

\usepackage[margin=1in]{geometry}

\usepackage{lmodern}
\usepackage{amsmath}
\usepackage{amssymb}
\usepackage{upgreek}
\usepackage{pgfplots}
\usepackage{qtree}
\usepackage{calc}
\usepackage[normalem]{ulem}
\usepackage[utf8]{inputenc}
\usepackage[T1]{fontenc}

\begin{document}

\section{Probabilidade}
A probabilidade é o estudo das experiências aleatórias. \\[10pt]
\uline{Espaço amostral}: denotado por $\Omega$, é o conjunto de \uline{todos} os resultados possíveis em uma experiência. \\[5pt]
\uline{Evento}: é um subconjunto de $\Omega$ ao qual é associado uma probabilidade em \uline{uma} amostragem aleatória.


\subsection{Famílias de Eventos}
Uma família de eventos é um conjunto de eventos. \\
A maior família de eventos é o conjunto potência de $\Omega$, denotado $\mathcal{P}(\Omega)$. \\[5pt]
Propriedades das famílias de eventos:
\begin{itemize}
  \item $\Omega \in \mathcal{F}$
  \item $A \in \mathcal{F} \implies A^c \in \mathcal{F}$
  \item $A, B \in \mathcal{F} \implies (A \cup B) \in \mathcal{F}$
\end{itemize}


\subsection{Função Probabilidade}
Uma função probabilidade é uma função do tipo $P:\, \mathcal{F} \to [0, 1]$. \\[5pt]
Propriedades de uma função probabilidade:
\begin{itemize}
  \item $0 \leq P(A) \leq 1$
  \item $P(\Omega) = 1$,\enspace $P(\varnothing) = 0$
  \item $P(A \cup B) = P(A) + P(B) - P(A \cap B)$
  \item $P(A^c) = 1 - P(A)$
  \item $A \subseteq B \implies P(A) \leq P(B)$
\end{itemize}


\subsection{Espaço de Probabilidade}
Um espaço de probabilidade é uma tripla da forma $(\Omega, \mathcal{F}, P)$. \\
O espaço é associado à uma experiência aleatória.

\subsubsection{Espaços Equiprováveis}
Um espaço de probabilidade $(\Omega, \mathcal{F}, P)$ é equiprovável quando
\[ \forall \, a, b \in \Omega: \> P({a}) = P({b}) = \frac{1}{\mid \Omega \mid} \]


\subsection{Probabilidade Clássica}
A função de probabilidade clássica para espaços \uline{equiprováveis} é
\[ P(A) = \frac{\mid A \mid}{\mid \Omega \mid} \]


\subsection{Probabilidade Condicional}
A probabilidade de um evento $B$ acontecer após um evento $A$ é
\[ P(B|A) = \frac{P(A \cap B)}{P(A)} \]



\end{document}
