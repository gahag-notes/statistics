\documentclass{article}

\usepackage[margin=1in]{geometry}

\usepackage{amsmath}
\usepackage{amssymb}
\usepackage{booktabs}
\usepackage{calc}
\usepackage{lmodern}
\usepackage{multirow,multicol}
\usepackage{pgfplots}
\usepackage{qtree}
\usepackage{upgreek}
\usepackage[normalem]{ulem}
\usepackage[utf8]{inputenc}
\usepackage[T1]{fontenc}
\usepackage[hidelinks]{hyperref}

\begin{document}

\section{Probabilidade}
A probabilidade é o estudo das experiências aleatórias.
\vspace{-5pt}
\begin{table}[h]
  \begin{tabular}{ll}
    \multicolumn{2}{l}{\hspace{-6pt}\textbf{Conceitos fundamentais}} \\ \midrule
     Experimento aleatório & Experimento cujo resultado não pode ser previsto com exatidão. \\[1pt]
     Espaço amostral ($\Omega$) & O conjunto de \uline{todos} os resultados possíveis em uma experiência. \\[1pt]
     Evento & Um subconjunto de $\Omega$. \\[1pt]
  \end{tabular}
\end{table} \vspace{-10pt}


\subsection{Função Probabilidade}
Uma função probabilidade é uma função do tipo $P:\, \mathcal{P}(\Omega) \to [0, 1]$. \\[5pt]
Propriedades de uma função probabilidade:
\begin{itemize}
  \item $P(\Omega) = 1$,\enspace $P(\varnothing) = 0$
  \item $0 \leq P(A) \leq 1$
  \item $\forall \: A_1, A_2, \hdots, A_n \text{ disjuntos}:\: P\left(\bigcup\limits_{i=1}^{n} A_i\right) = \sum\limits_{i=1}^{n} P(A_i)$
  \item $P(A^c) = 1 - P(A)$
  \item $A \subseteq B \implies P(A) \leq P(B)$
  \item $P(A \cup B) = P(A) + P(B) - P(A \cap B)$
\end{itemize}


\subsection{Probabilidade Clássica}
A função de probabilidade clássica para espaços \uline{equiprováveis} é
\[ P(A) = \frac{\mid A \mid}{\mid \Omega \mid} \]


\subsection{Probabilidade Condicional}
A probabilidade de um evento $B$ acontecer dado um evento $A$ é
\[ P(B|A) = \frac{P(A \cap B)}{P(A)} \]
\uline{Teorema de Bayes}:
\[ P(B|A) = \frac{P(B)}{P(A)} \cdot P(A|B) \]


\subsection{Probabilidade Total}
O conjunto de eventos $\{ A_1, \hdots, A_n \}$ forma uma \uline{partição} de $\Omega$ se e somente se
\begin{itemize}
  \item $\forall \, i \neq j: A_i \cap A_j = \varnothing$ \quad (Os eventos são disjuntos entre si)
  \item $\bigcup\limits_{i=1}^{n} A_i = \Omega$
\end{itemize}
\uline{Teorema da probabilidade total}: Dada uma partição $\{ A_1, \hdots, A_n \}$
\[ P(B) = \sum_{i = 1}^{n} P(A_i) \cdot P(B|A_i) \]


\subsection{Independência}
Dois eventos $A$ e $B$ são independentes se e somente se
\[ P(A \cap B) = P(A) \cdot P(B) \]
\uline{Corolário}: Se $A$ e $B$ são independentes, então $P(A|B) = P(A)$. \\[10pt]
Isto significa que a ocorrência de um evento não afeta a probabilidade do outro numa mesma amostra. \\
Portanto, eventos disjuntos \uline{não são independentes}, pois não podem ocorrer simultaneamente. \\[10pt]
Propriedade: Se $A$ e $B$ são independentes, então os seguintes conjuntos são independentes entre si:
\begin{itemize}
  \item $A$ e $B^c$
  \item $A^c$ e $B$
  \item $A^c$ e $B^c$
\end{itemize}

%% old --------------------------------------------------------------------
\hrule

\subsection{Famílias de Eventos}
Uma família de eventos é um conjunto de eventos. \\
A maior família de eventos é o conjunto potência de $\Omega$, denotado $\mathcal{P}(\Omega)$. \\[5pt]
Propriedades das famílias de eventos:
\begin{itemize}
  \item $\Omega \in \mathcal{F}$
  \item $A \in \mathcal{F} \implies A^c \in \mathcal{F}$
  \item $A, B \in \mathcal{F} \implies (A \cup B) \in \mathcal{F}$
\end{itemize}


\subsection{Espaço de Probabilidade}
Um espaço de probabilidade é uma tripla da forma $(\Omega, \mathcal{F}, P)$. \\
O espaço é associado à uma experiência aleatória.

\subsubsection{Espaços Equiprováveis}
Um espaço de probabilidade $(\Omega, \mathcal{F}, P)$ é equiprovável quando
\[ \forall \, a, b \in \Omega: \> P({a}) = P({b}) = \frac{1}{\mid \Omega \mid} \]


\subsection{Variáveis Aleatórias}
Uma variável aleatória é uma função que associa alguma propriedade de dado resultado à um número real.
\[ X: \Omega \to \mathbb{R} \]
A probabilidade de uma variável aleatória assumir um valor $c \in \mathbb{R}$ é
\[ P(X = c) \]
Se a imagem de $X$ é um conjunto enumerável, dizemos que esta variável é \uline{discreta}. \\[5pt]
Se a imagem de $X$ é um conjunto não enumerável, dizemos que esta variável é \uline{contínua}. \\[5pt]

\subsubsection{Distruibuição}
A distribuição de uma variável aleatória é definida por
\[ P(X = x_k) \:\big|\: k \in \mathbb{N}^* \]
Propriedades:
\begin{itemize}
  \item $0 \leq P(X = x_k) \leq 1$
  \item $\sum\limits_k P(X = x_k) = 1$
\end{itemize}
% example here: distribuição uniforme, bernoulli, binomial, poisson, geométrica
.


\subsection{Densidade}
A probabilidade de uma variável \uline{contínua} estar em um invervalo $[a,b]$ é
\[ P(a \leq X \leq b) = \int_b^a f(x) dx, \qquad f: \mathbb{R} \to \mathbb{R}^+ \]
onde $f$ é a função de densidade de probabilidade associada a $X$. \\[10pt]
Uma função de densidade de probabilidade satisfaz as propriedades:
\begin{itemize}
  \item $\forall\, x \in \mathbb{R}: f(x) \geq 0$.
  \item $\int\limits_{-\infty}^{\infty} f(x) dx = 1$.
\end{itemize}
\uline{Teorema}: Para qualquer distribuição, a probabilidade de uma variável aleatória estar em um intervalo é
\[ P(m \leq x \leq n) = f(n) - f(m) \]

\subsubsection{Distribuição Uniforme}
A função densidade para uma ditribuição uniforme é
\[
  f(x) = \begin{cases}
          \dfrac{1}{\omega} \quad 0 \leq x \leq \omega \\[10pt]
          0 \quad \text{caso contrário}
         \end{cases}
\]
Já para uma distribuição uniforme reduzida a um intervalo $[a,b] \subset \mathbb{R}$
\[
  f(x) = \begin{cases}
          \dfrac{1}{b - a} \quad a \leq x \leq b \\[10pt]
          0 \qquad \text{caso contrário}
         \end{cases} \\[10pt]
\]

\pagebreak

\subsection{Esperança Matemática}
Dada uma variável aleatória $X$, a esperança matemática de $X$ é
\begin{align*}
  & E(X) = \sum_{k=1}^{\infty} x_k \cdot P(X = x_k) && \text{$X$ discreto.} \\[5pt]
  & E(X) = \int_{-\infty}^{\infty} x \cdot f(x) \: dx && \text{$X$ contínuo.}
\end{align*}
Propriedades:
\begin{itemize}
  \item $\forall\: a \in \mathbb{R}: E(a) = a$
  \item $\forall\: a \in \mathbb{R}: E(aX) = a \cdot E(X)$
  \item $E(X + Y) = E(X) + E(Y)$
  \item $E(X^2) \geq {E(X)}^2$
\end{itemize}

\subsection{Variância}
A variância de uma variável aleatória $X$ é
\begin{align*}
  \text{var}(X) & = {E(X - E(X))}^2 \\
  & = E(X^2) - {E(X)}^2
\end{align*}
A covariância de duas variáveis aleatórias $X$ e $Y$ é
\[ \text{cov}(X, Y) = E(XY) - E(X) \cdot E(Y) \]
Propriedades:
\begin{itemize}
  \item $\forall\: a \in \mathbb{R}: \text{var}(a) = 0$
  \item $\forall\: a \in \mathbb{R}: \text{var}(aX) = a^2 \cdot \text{var}(X)$
  \item $\text{var}(X + Y) = \text{var}(X) + \text{var}(Y) + 2 \cdot \text{cov}(X, Y)$
  \item $X$ e $Y$ independentes $\implies \text{cov}(X,Y) = 0 \implies \text{var}(X + Y) = \text{var}(X) + \text{var}(Y)$
\end{itemize}

\subsection{Desvio Padrão}
O desvio padrão de uma variável aleatória $X$ é
\begin{align*}
  \text{dp}(X) & = \sqrt{\text{var}(X)} \\
  & = \sqrt{{E(X - E(X))}^2}
\end{align*}

\pagebreak

\subsection{Densidade Normal Geral}
\begin{gather*}
  X \sim N(\mu, \sigma^2) \\[5pt]
  f(x) = \frac{1}{\sqrt{2 \pi \sigma^2}} \cdot \exp \left(- \frac{{(x - u)}^2}{2\sigma^2} \right)
\end{gather*}



\end{document}
